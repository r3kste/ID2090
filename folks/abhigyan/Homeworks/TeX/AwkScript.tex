\documentclass{article}
\usepackage{listings}
\usepackage{color}
\usepackage{graphicx}
\graphicspath{{/home/abhigyan/Pictures/}}

\definecolor{darkgreen}{rgb}{0,0.6,0}
\definecolor{gray}{rgb}{0.5,0.5,0.5}
\definecolor{mauve}{rgb}{0.58,0,0.82}

\lstset{
  frame=tb,
  language=awk,
  aboveskip=2mm,
  belowskip=2mm,
  showstringspaces=false,
  columns=flexible,
  basicstyle={\small\ttfamily},
  numbers=none,
  numberstyle=\tiny\color{gray},
  keywordstyle=\color{blue},
  commentstyle=\color{darkgreen},
  stringstyle=\color{mauve},
  breaklines=true,
  breakatwhitespace=true,
  tabsize=4
}

\title{Homework 4}
\author{Abhigyan Chattopadhyay
  \\ME19B001
  \\IIT Madras
}

\begin{document}
\maketitle

\pagebreak
\section{Homework - Session 9}
\subsection{Code from awk script}

\begin{lstlisting}
#!/usr/bin/gawk
BEGIN{
  	#Setting the file separator as ,
  	FS=",";
	avg1=0;
	avg2=0;
	avg3=0;
	avg4=0;
	count=0;
}
{
	x = $1+$2+$3;
	#printing the value of the sum just after the input data with a comma
	print($0,",",x);
	#currently, just finding the summation of each line, average will be taken later
	avg1=avg1+$1;
	avg2=avg2+$2;
	avg3=avg3+$3;
	avg4=avg4+x;
	count++;
}
END{
  	#taking the averages
	avg1/=count;
	avg2/=count;
	avg3/=count;
	avg4/=count;
	print("Averages:");
	print(avg1,",", avg2,",", avg3,",", avg4);
	print("Lines parsed = ", count);
}
\end{lstlisting}

\begin{figure}
  \includegraphics[width=\linewidth]{Awk-Script-0}
  \caption{The starting few lines of the contents of the csv file that contains the comma-separated values of 1500 numbers, their squares, and their cubes}
  \label{fig:pic1}
\end{figure}


\begin{figure}
  \includegraphics[width=\linewidth]{Awk-Script-1}
  \caption{The starting few lines of the output of the awk script when run on the csv file}
  \label{fig:pic2}
\end{figure}


\begin{figure}
  \includegraphics[width=\linewidth]{Awk-Script-2}
  \caption{The ending few lines of the output of the awk script}
  \label{fig:pic3}
\end{figure}

\end{document}
