\documentclass[12pt,a4paper]{article}
\title{Homework 6}
\author{Abhigyan Chattopadhyay \\
	ME19B001
}
\usepackage{dirtytalk}
\usepackage[a4paper, margin=1.5cm]{geometry}
\usepackage{listings}
\usepackage{hyperref}
\usepackage{graphicx}
\usepackage{subcaption}
\graphicspath{{/home/abhigyan/Pictures/}}
\hyphenpenalty=10000
\hbadness=10000
\begin{document}
\maketitle
\pagebreak

\section{Homework - Session 13}

\subsection{Question 1: List out all the built-in data types of C and C++ language, the space they consume in the
memory and their ability to represent information.}

\subsubsection{Data Types in C}

\begin{tabular}{|l|r|p{9cm}|}
	\hline
	Type & Size (in bits) & Type of Data stored\\
	\hline 
	signed char & 8 & Integers (-128 to 127)\\
	unsigned char & 8 & Integers (0 to 255)\\
	char & 8 & Characters (ASCII) (Actually Integers allocated to ASCII Code)\\
	short int & 16 & Integers (-32768 to 32767)\\
	unsigned short int & 16 & Integers (0 to 65535)\\
	int & 32 & Integers (-2,147,483,648 to 2,147,483,647)\\
	unsigned int & 32 & Integers (0 to 4,294,967,295)\\
	long int & 32 or 64 & Integers (-2,147,483,648 to 2,147,483,647 or -9,223,372,036,854,775,808 to 9,223,372,036,854,775,807)\\
	unsigned long int & 32 or 64 & Integers (0 to 4,294,967,295 or 0 to 18,446,744,073,709,551,615)\\
	long long int & 64 & Integers (-9,223,372,036,854,775,808 to 9,223,372,036,854,775,807)\\
	unsigned long long int & 64 & Integers (0 to 18,446,744,073,709,551,615)\\
	float & 32 & Real Numbers (1.2E-38 to 3.4E+38)\\
	double & 64 & Real Numbers (2.3E-308 to 1.7E+308)\\
	long double & 80 & Real Numbers (3.4E-4932 to 1.1E+4932)\\
	\hline
\end{tabular}

\subsubsection{Data Types in C++}

\begin{tabular}{|l|r|p{9cm}|}
	\hline
	Type & Size (in bits) & Type of Data stored\\
	\hline
	bool & 1 & True or False value (0 or 1)\\
	signed char & 8 & Integers (-128 to 127)\\
	unsigned char & 8 & Integers (0 to 255)\\
	char & 8 & Characters (ASCII) (Actually Integers allocated to ASCII Code)\\
	wchar\_t & 16 or 32 & Characters (Unicode or other ASCII Extensions)\\
	short int & 16 & Integers (-32768 to 32767)\\
	unsigned short int & 16 & Integers (0 to 65535)\\
	int & 32 & Integers (-2,147,483,648 to 2,147,483,647)\\
	unsigned int & 32 & Integers (0 to 4,294,967,295)\\
	long int & 32 or 64 & Integers (-2,147,483,648 to 2,147,483,647 or -9,223,372,036,854,775,808 to 9,223,372,036,854,775,807)\\
	unsigned long int & 32 or 64 & Integers (0 to 4,294,967,295 or 0 to 18,446,744,073,709,551,615)\\
	long long int & 64 & Integers (-9,223,372,036,854,775,808 to 9,223,372,036,854,775,807)\\
	unsigned long long int & 64 & Integers (0 to 18,446,744,073,709,551,615)\\
	float & 32 & Real Numbers (1.2E-38 to 3.4E+38)\\
	double & 64 & Real Numbers (2.3E-308 to 1.7E+308)\\
	long double & 80 & Real Numbers (3.4E-4932 to 1.1E+4932)\\
	\hline
\end{tabular}

References: 

\begin{enumerate}
	\item \href{https://www.gnu.org/software/gnu-c-manual/gnu-c-manual.html}{GNU C Manual}
	\item \href{https://www.geeksforgeeks.org/c-data-types/}{C - Geeks for Geeks}
\end{enumerate}

\subsection{Question 2: Are there built in data types in Fortran that are not in C? Check out the complex number, for
example and write about that.}

C does not have the logical or the complex data type which are built-in for Fortran. However, newer versions of C have these in them as well, in the standard headers. C++ has a built-in boolean data type, but doesn't have a Complex data type either. This basically shows that Fortran is much more oriented towards scientific coding than C/C++ which are more general purpose languages.

\subsection{Question 3: From what you can pick from Wiki pages, list out the built in data types in the languages
Python, Sage and Octave. Comment on what needs to be programmed if you were to build up such
a functionality in, say, C language.}

\subsubsection{Data Types in Python/Sage:}

Sage is basically built on top of multiple libraries, and at its base, it is purely Python. As a result, Python and Sage have the same built-in or primitive data types. On top of this, Sage adds multiple derived or user-defined data types (like VectorSpace, Matrix, MatrixSpace, etc.) Thus, I'm only listing the Python data types here, as both Python and Sage have too many derived data types to list out explicitly here. To know more about the, see the \href{http://doc.sagemath.org/html/en/tutorial/programming.html#data-types}{Sage Documentation} and the \href{https://docs.python.org/3/library/datatypes.html}{Python Documentation}.

\begin{tabular}{|l|p{5cm}|p{6cm}|}
	\hline	
	Type & Sub-Types & Purpose\\
	\hline
	Numeric & int, float, complex & Represent any numerical value\\
	Boolean & bool & True/False Values\\
	Sequences & str, list, tuple & Groups of ordered items\\
	Dictionary & dict, set & Groups of unordered items, linked lists\\
	\hline
\end{tabular}

\begin{figure}
	\centering
	\includegraphics[width=0.8\textwidth]{Python_Work}
	\caption{One of each in-built data type in Pure Python}
\end{figure}

\begin{figure}
	\centering
	\fbox{\begin{subfigure}[b]{0.47\linewidth}
		\includegraphics[width=\linewidth]{Sage_Work_1}
		\caption{One of each of the Python built-in data types in Sage. Note how Sage stores the numerical data types as types other than the Python primitives}
	\end{subfigure}}
	\fbox{\begin{subfigure}[b]{0.47\linewidth}
		\includegraphics[width=\linewidth]{Sage_Work_2}
		\caption{Some of the other (cooler) data types in Sage. Expression is the most commonly encountered one in symbolic computations}
	\end{subfigure}}
	\caption{Various data types in Sage}
\end{figure}

\pagebreak
\subsubsection{Data Types in Octave:}

In Octave, every single piece of information is a matrix. The contents of the matrix can be of different data types, which are tabulated below:

\begin{tabular}{|l|p{5cm}|p{6cm}}
	\hline
	Data Types & Use Case\\
	\hline
	int8, int16, int32, int64 & Integers\\
	uint8, uint16, uint32, uint64 & Unsigned Integers\\
	char & Characters\\
	double, single & Real Numbers\\
	logical & Boolean\\
	\hline
\end{tabular}

Also, see the \href{https://octave.org/doc/v4.2.0/Built_002din-Data-Types.html#Built_002din-Data-Types} {Octave Documentation}.

\subsubsection{Difficulties in implementing this in C}

One of the biggest disadvantages of C is that it doesn't support operator overloading. This leads to many major bottlenecks during the implementation of various functionality that are available in Python, Sage and Octave. As a result, we would need to create structs and implement each of them as individual constructs, and still not have the notational simplicity available in Python, Sage and Octave.

Thus, to build up such a functionality in C language, we would need to create various structs that could contain multiple numerical and non-numerical values in them, depending on what they were created for. However, the only way to make them work like their Sage or Octave counterparts would be to call functions. For example, while adding two matrices in Octave, we can use: \lstinline{x = A+B;} or \lstinline{x=A*B} where both A and B are matrices.

In C, it might be implemented as: \lstinline{x = A.add(B)} or \lstinline{x = A.matrix_product(B)} which loses out the intuitive meaning that is conveyed in the former language. However, C is a blazing fast language, and anything that can be accomplished in Sage or MATLAB/Octave can be also achieved in C, albeit via a longer and more tedious code writing process. This is the reason why MATLAB/Octave and Sage might be used for prototyping and modelling, while C is mostly used for production.

\pagebreak

\subsection{Question 4: After you have installed jupyter-notebook and python, launch the notebook and watch out which port number the notebook server is running on. Can you also find out the process ID of this notebook server while it is running?}

\begin{figure}[h]
	\includegraphics[width=0.7\linewidth]{Port_ID}
	\caption{Port Number is 8888, gleaned from address bar}
\end{figure}
\begin{figure}[h]
	\includegraphics[width=0.7\linewidth]{Process_ID}
	\caption{Process ID is 16546, gleaned from bash command}
\end{figure}

\pagebreak

\subsection{Question 5: From Wiki pages, list out what are the different languages that are supported by the jupyter notebook interface. Figure out what is meant by ``markup syntax used in html or xml'' and the ``mark down'' syntax of jupyter notebook.}

\subsubsection{Jupyter Notebooks}
Quoting verbatim from the \href{https://jupyter4edu.github.io/jupyter-edu-book/jupyter.html#targetText=The%20Jupyter%20system%20supports%20over,additional%20kernels%20may%20be%20installed.}{Jupyter Edu Book}:

\say{The Jupyter system supports over 100 programming languages (called “kernels” in the Jupyter ecosystem) including Python, Java, R, Julia, Matlab, Octave, Scheme, Processing, Scala, and many more. Out of the box, Jupyter will only run the IPython kernel, but additional kernels may be installed. Language support continues to be added by the open source community and the best source for an up-to-date list is the wiki page maintained by the project: \href{https://github.com/jupyter/jupyter/wiki/Jupyter-kernels}{Jupyter Kernels}. These projects are developed and maintained by the open source community and exist in various levels of support. Some kernels may be supported by a vast number of active (and even paid) developers, while others may be a single person’s pet project. When trying out a new kernel, we suggest exploring a kernel’s community of users and documentation to see if it has an appropriate level of support for your (and your students’) use.}

Thus, some languages supported in Jupyter (which we students of MM2090 have heard of in the last couple of weeks) are:

\begin{itemize}
	\item Python
	\item Julia
	\item Sage
	\item MATLAB/Octave
\end{itemize}

\subsubsection{Markup vs Markdown}

Markup is the name given to any language that uses tags to define elements in a document. It is a generic term for a number of languages, like Hyper Text Markup Languages (HTML) and XML (eXtensible Markup Language). 

Markdown, on the other hand, is a specific language, which is used to generate various types of markup depending on the scenario. In a browser, Markdown produces HTML, which is then rendered by the browser. \LaTeX{} can also be used within Markdown, and Markdown can basically be read without needing to be rendered, as it focuses on readability. There are some flavours of Markdown, most notably the GitHub flavour (used in many GitHub pages in the Readme.md file).

\pagebreak
\section{Homework - Session 14}

\subsection{Question 1: Edit the jupyter notebook to explore a mapping of all data types of C language with the corresponding ones of python.}

\begin{tabular}{|l|l|}
	\hline	
	C & Python\\
	\hline
	(unsigned and signed) short int, int, long int, long long int & int\\
	float, double, long double & float\\
	(unsigned and signed) char & str\\
	\hline
\end{tabular}

\begin{figure}[h]
	\includegraphics[scale=0.4]{Jupyter_Work}
	\caption{Trying out various Data Types in Python}
\end{figure}

\subsection{Question 2: Figure out what data types are available in python that are not readily available from C.}

As it has already been seen earlier, we know that C doesn't have any built in complex data type. Neither does it have dicts, which are available in Python. (Note: C \textit{does} have an array data type that closely corresponds to Python's \lstinline{list}, and by extension, \lstinline{char*} is similar to Python's \lstinline{str}. Thus, we can't say that those are not reaily available in C, as they are available in \lstinline{<stdio.h>} itself.)

\end{document}