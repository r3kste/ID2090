\documentclass[12pt,a4paper]{article}
\usepackage{graphicx}
\usepackage{hyperref}
\title{Summary of \LaTeX{}}
\author{Abhigyan Chattopadhyay\\
ME19B001}
\setcounter{section}{-1}

\begin{document}
\maketitle
\section{Introduction}
\LaTeX{} is a document preparation system, which results in very high quality text output. Unlike Microsoft Office, or LibreOffice, or any of the document editing suites, \LaTeX{} is not a WYSIWYG (what you see is what you get) editor, and writing a document in \LaTeX{} is more like coding a program. This brings in two things:

\begin{enumerate}
\item Uniformity in document looks, and
\item Abstraction and automation of unimportant tasks like text alignment and font sizes
\end{enumerate}

Both of these tasks invariably take up a large amount of our productivity in WYSIWYG editors, hence \LaTeX{} is used a lot in scientific publications and is considered a de-facto standard. Here's the \href{https://www.latex-project.org/}{\LaTeX{}} website.

In this document, I plan to show how to make the following:

\begin{enumerate}
\item Numbered and Bulleted lists
\item Tables
\item Equations
\item Citations (using bibtex)
\item Hyperlinks (using hyperref)
\item Figures (using graphicx)
\end{enumerate}

\section{Numbered and Bulleted Lists}

\subsection{Numbered List}
An example numbered list:

\fbox{
\begin{minipage}{15em}

\textbackslash begin\{enumerate\}

\textbackslash item Item 1

\textbackslash item Item 2

\textbackslash item Item 3

\textbackslash end\{enumerate\}
\end{minipage}
}

This is rendered in \LaTeX{} as:

\begin{enumerate}
\item Item 1
\item Item 2
\item Item 3
\end{enumerate}

\subsection{Bulleted List}
An example bulleted list:

\fbox{
\begin{minipage}{15em}

\textbackslash begin\{itemize\}

\textbackslash item Item 1

\textbackslash item Item 2

\textbackslash item Item 3

\textbackslash end\{itemize\}
\end{minipage}
}

This is rendered in \LaTeX{} as:

\begin{itemize}
\item Item 1
\item Item 2
\item Item 3
\end{itemize}
\pagebreak
\section{Tables}

An example table:

\fbox{
\begin{minipage}{35em}

\textbackslash begin\{tabular\}\{l r$|$c$|$p\{2cm\}$|$\}

\textbackslash hline

Column left \& Column right \& Column centre \& Column Fixed Width \textbackslash\textbackslash

\textbackslash hline

Text \& Number \& Anything \& Fixed width data \textbackslash\textbackslash

\textbackslash hline

Cats \textbackslash\& Dogs \& 10 \& Meow \& Woof \textbackslash\textbackslash

Cows \& Sheep (No \textbackslash textbackslash hline above) \& 9 \& Moo \& Baa \textbackslash\textbackslash

\textbackslash hline

\textbackslash end\{tabular\}
\end{minipage}
}

This is rendered in \LaTeX{} as:

\begin{tabular}{|l r|c|p{2cm}|}
\hline
Column left & Column right & Column centre & Column Fixed Width \\
\hline
Text & Number & Anything & Fixed width data \\
\hline
Cats \& Dogs & 10 & Meow & Woof \\
Cows \& Sheep (No \textbackslash hline above) & 9 & Moo & Baa \\
\hline
\end{tabular}

\section{Equations}

To insert an equation in \LaTeX{}, we surround it by \$ \$ to make an inline equation, while we surround it with \$\$ \$\$ to make it a separate equation.

To make it a numbered and referable equation, we need to use the \textbackslash equation environment.

For example:

\fbox{
\begin{minipage}{35em}

\end{minipage}
}

\end{document}